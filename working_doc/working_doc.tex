\documentclass[12pt,reqno, a4paper]{amsart}



% symbols and fonts 
\usepackage{amsmath}
\usepackage{amsfonts}
\usepackage{amssymb}
\usepackage{mathtools}
\usepackage{tikz-cd}

% ordering and geometry
\usepackage{amsthm}
\usepackage{float}
\usepackage{geometry}

% quality of life 
\usepackage[ color = blue!20]{todonotes}
\usepackage{hyperref}
\usepackage[capitalise,noabbrev]{cleveref}
\usepackage{mathtools}
\usepackage{graphicx}
\graphicspath{ {./images/} }

% choosing font
\usepackage{libertine}
\usepackage[T1]{fontenc}


\setcounter{section}{0} %starts on section 1
\numberwithin{equation}{section}
\theoremstyle{definition}
\newtheorem{theorem}{Theorem}[section]
\newtheorem{definition}[theorem]{Definition}
\newtheorem{proposition}[theorem]{Proposition}
\newtheorem{lemma}[theorem]{Lemma}



% some quality-of-life commands 

% matrix commands 
\newcommand{\tr}{\mathrm{tr}}
% blackboard bolds 
\newcommand{\C}{\mathbb{C}}


\title{Formalising $\det(e^A) = e^{\tr(A)}$ in Lean} 
\author{}
\date{}


\begin{document}
\maketitle

\section{The non-formal proof}

We list some facts that the proof follows from. The matrix $A$ is assumed to have entries in $\C$ and $P$ is an invertible matrix with entries in $\C$. 

\begin{enumerate}
	\item For all $M,P$, we have $\det(e^M) = \det(Pe^MP^{-1})$.
	\item For every matrix $A$, there is an invertible $P$ such that $A = PUP^{-1}$, with $U$ upper-triangular.
	\item $\det(U) = \prod (U)_{(i,i)}$.
	\item If $U$ is upper-triangular, then $(e^U)_{(i,i)} = e^{(U)_{(i,i)}}$. 
	\item For complex numbers $a$ and $b$, we have $e^{a+b} = e^a e^b$.
	\item For all $M,P$, we have $\tr(PMP^{-1}) = \tr(M)$.
	\item For all $M, P$, we have $\det(PMP^{-1}) = \det(M)$. 
\end{enumerate}

The proof now follows in the following argument: 
\begin{proof}
	\begin{align*}
		\det(e^A) & = \det(Pe^UP^{-1}) & \text{by 1 and 2} \\ 
		& = \det(e^U) & \text{by 7} \\
		& = \prod e^{(U)_{(i,i)}} & \text{by 3 and 4} \\
		& = e^{\sum U_{(i,i)}} & \text{by 5} \\
		& = e^{\tr(U)} & \text{by definition of trace} \\ 
		& = e^{\tr(PUP^{-1})} & \text{by 6} \\ 
		& = e^{\tr(A)} & \text{by 2 again}
	\end{align*}
\end{proof}

We list the difficulties of the various parts. 

\begin{enumerate}
	\item This already exists in Lean, in \texttt{Matrix.exp\_conj}.
	\item This will need proving. This can be done by showing that no minimal counterexample exists, using the fact that every matrix over with entries in $\C$ has an eigenvector.
	\item This can be proven inductively using \texttt{Matrix.det\_succ\_column}.
	\item This will need proving.
	\item This should follow from \texttt{Matrix.trace\_mul\_cycle}, since applying one cyclic permutation allows us to cancel $P$ with its inverse.
	\item This should be in Lean (what is it called?).
\end{enumerate}







\end{document}
